\begin{abstract}
Με την σταδιακή μετακίνηση της επεξεργασίας των δεδομένων και την
εκτέλεση των εφαρμογών από τοπικά υπολογιστικά συστήματα σε
εικονικοποιημένα \en{cloud} συστήματα, γίνεται εμφανές πως τα συμβατικά
λειτουργικά συστήματα αδυνατούν να επιτελέσουν αποδοτικά τον ρόλο τους.
Σχεδιασμένο δεκαετίες πριν, ένα συμβατικό λειτουργικό σύστημα εισάγει
περιττές εξαρτήσεις, αυξημένες ανάγκες σε υπολογιστικούς πόρους και
χρονικές επιβαρύνσεις κατά την \en{cloud} εκτέλεση του. Η τεχνολογία
των \en{unikernel}, στοχεύει στην εξάλειψη των ανώτερων προβλημάτων,
δημιουργώντας μια ελαφριά, εξειδικευμένη και ταχύτατη εικονική
μηχανική, η οποία στοχεύει στην εκτέλεση μιας μοναδικής λειτουργίας,
και που ταιριάζει καλύτερα στο \en{cloud περιβάλλον}. Ταυτόχρονα, η
εικονικοποίηση των υπολογιστικών συστημάτων οδηγεί αναπόφευχτα σε υποβέλτιστη
χρησιμοποίηση των διαθέσιμων πόρων, και ειδικά της μνήμης. Ο
μηχανισμός ελαστικής μνήμης \en{utmem} εκμεταλλεύεται την συνεργασίας
του εικονικοποιημένου συστήματος και του επόπτη, επιτρέποντας
την παραεικονοποίηση, με σκοπό την αυξημένη απόδοση χρήσης της
μνήμης από τις εικονικοποιημένες εφαρμογές. Σκοπός της παρούσας
εργασίας, είναι η μελέτη των χαρακτηριστικών και της φιλοσοφίας
διάφορων \en{unikernel} περιβάλλοντων, καθώς και του μηχανισμού της
\en{utmem} με τελικό στόχο την ενσωμάτωση του τελευταίου σε κάποιο
\en{unikernel} περιβάλλον. Τέλος, αποτιμάται πειραματικώς η χρηστική αξία του εν λόγω συνδυασμού
των δύο αυτών καινοτόμων τεχνολογιών, ως προς συμβατικά περιβάλλοντα
εικονικοποίησης και περιβάλλοντα \en{lightweight} εικονικοποίησης.
\end{abstract}

\newpage
\renewcommand{\abstractname}{\en{Abstract}}

\begin{abstract}
\en{
The gradual transition of data processing and application execution, from local
computer system to virtualized cloud systems, makes apparent that conventional
operating systems are unable to fulfill their role effectively. Designed
decades ago, a conventional operating system introduces superfluous
dependencies, increased needs in computational resources and temporal
delays when executed on the cloud. The technology of unikernel aims to the
elimination of the aforementioned problems, by creating a lightweight,
specialized and fast virtual machine, which supports execution of a unique
function, and which fits better on the cloud environment. Meanwhile, the
virtualization of computer system inevitably leads to suboptimal usage of the available
resources, especially that of memory. The elastic memory mechanism utmem
takes advantage of the cooperation between the virtualized system and the
hypervisor, allowing paravirtualization techniques, in purpose to elevate
efficiency of memory used by virtualized applications. The goal of this
thesis is the study of the philosophy and the characteristics of various
unikernel frameworks, as well as the mechanism of utmem, with ultimate
cause the incorporation of the latter to a unikernel environment. Finally,
the functional value of that innovative technologies' combination is
evaluated, in aspect of both conventional virtualization environments and
lightweight virtualization environments.
}
\end{abstract}
